\documentclass[conference]{IEEEtran}
\usepackage{cite}
\usepackage{amsmath,amssymb,amsfonts}
\usepackage{algorithmic}
\usepackage{graphicx}
\usepackage{textcomp}
\usepackage[dvipsnames]{xcolor}
\usepackage{tikz}
\usepackage{pgfplots}\pgfplotsset{compat=1.17}
\usetikzlibrary{arrows.meta,positioning,calc}
\usepackage{hyperref}
\usepackage{csquotes}
\usepackage{nicefrac}

\def\BibTeX{{\rm B\kern-.05em{\sc i\kern-.025em b}\kern-.08em
    T\kern-.1667em\lower.7ex\hbox{E}\kern-.125emX}}

\pgfplotsset{select coords between index/.style 2 args={
    x filter/.code={
        \ifnum\coordindex<#1\def\pgfmathresult{}\fi
        \ifnum\coordindex>#2\def\pgfmathresult{}\fi
    }
}}

% correct bad hyphenation here
\hyphenation{op-tical net-works semi-conduc-tor si-mulando}

\usepackage{subfiles}

\setlength{\textfloatsep}{5pt}

\begin{document}

\title{Análisis de Transistores. El Diodo en Frecuencia.}
\author{
\IEEEauthorblockN{Alejandro Nahuel Heir}
\IEEEauthorblockA{  Instituto Tecnológico de Buenos Aires\\
                    Buenos Aires, Argentina\\
                    aheir@itba.edu.ar}
\and
\IEEEauthorblockN{Matías Ezequiel Alvarez}
\IEEEauthorblockA{  Instituto Tecnológico de Buenos Aires\\
                    Buenos Aires, Argentina\\
                    matalvarez@itba.edu.ar}
}

\maketitle

\begin{abstract}
Es común en ámbitos ingenieriles utilizar modelos idealizados de diversos componentes electrónicos para simplificar cálculos y predicciones analíticas. En este trabajo se comparan los resultados teóricos con los obtenidos a partir de simulaciones en el software \emph{LTSpice} de distintos dispositivos semiconductores: diodo, BJT, JFET y MOSFET. Adicionalmente se evalúan polarizaciones y modelos de pequeña señal, profundizando en el del BJT y en el rango operativo de frecuencia del diodo.
\end{abstract}

\begin{IEEEkeywords}
diodo, transistor, BJT, JFET, MOSFET, punto Q, curva característica, pequeña señal.
\end{IEEEkeywords}

\section{Introducción}
\subfile{sections/intro}

\section{Diodo: 1N4148}
\subfile{sections/diodo-1n4148}

\section{NPN BJT: BC547}
\subfile{sections/bjt-bc547}

\section{N-Channel JFET: MPF102}
\subfile{sections/jfet-mpf102}

\section{N-Channel Enhancement
Mode MOSFET: NDS7002}
\subfile{sections/mosfet-nds7002}

\section{N-Channel Depletion Mode MOSFET: LND250}
\subfile{sections/mosfet-lnd250}

\section{Conclusión}
\subfile{sections/conclusion}

\begin{thebibliography}{1}

\bibitem{1n4148}
ON Semiconductor, \textquote{Small Signal Diode - 1N91x, 1N4x48, FDLL914, FDLL4x48}, 1N914/D, 2002 [Revised Aug. 2021].

\bibitem{bc547b}
NXP Semiconductors, \textquote{BC847-BC547 series
45 V, 100 mA NPN general-purpose transistors}, BC847\_BC547\_SER\_7, 2008 [Revised Dec. 2008].

\bibitem{mpf102}
ON Semiconductor, \textquote{MPF102 - JFET VHF Amplifier N-Channel - Depletion}, MPF102/D, 2006 [Revised Jan. 2006].

\bibitem{nds7002}
ON Semiconductor, \textquote{N-Channel Enhancement Mode Field Effect Transistor - 2N7000, 2N7002, NDS7002A}, NDS7002A/D, 1998 [Revised Jan. 2022].

\bibitem{infineon-note}
V. Barkhordarian, \textquote{Power MOSFET Basics}, International Rectifier, 1997.

\bibitem{lnd250}
Microchip, \textquote{LND150/LND250 N-Channel Depletion-Mode DMOS FETs}, DS20005454A, 2018 [Revised Aug. 2018].

\end{thebibliography}


\end{document}
