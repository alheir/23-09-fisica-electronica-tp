\documentclass[../main.tex]{subfiles}

\begin{document}
Las simulaciones de los componentes permitieron una comparación con la teoría y las hojas de datos de los distintos fabricantes.

Se obtuvo el punto de operación de los distintos dispositivos y se identificó la polarización de los mismos. A partir de ellos, se calcularon los parámetros de pequeña señal para cada uno, comparándolos con los especificados con los fabricantes, teniendo coincidencias y discrepancias. En particular, con el análisis de pequeña de señal del BJT, se comprobaron limitaciones en cuanto a lo \emph{pequeña} de dicha señal, límites por fuera de los cuales se distorsionaba la señal de salida. Con el análisis en frecuencia del diodo se comprobó el efecto de capacidades parásitas, las cuales alteran completamente el funcionamiento esperado del dispositivo.

Como observación final, todas las simulaciones realizadas en \emph{LTSpice} consideraban una temperatura ambiente estándar de $300K$, lo cual aseguraba un comportamiento térmico estable.
\end{document}