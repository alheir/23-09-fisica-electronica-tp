\documentclass[../main.tex]{subfiles}

\begin{document}
El objetivo del informe es analizar los dispositivos estudiados en Física Electrónica (23.09). Con dicho fin en mente, para cada componente se obtuvo en primera instancia el punto $Q$ de polarización para luego calcular los parámetros del modelo de pequeña señal para ese punto. Luego se graficaron las distintas curvas de entrada y salida de los dispositivos para lograr una comparación cualitativa de los resultados teóricos con los simulados. Se realizó, también, un análisis detallado del modelo de pequeña señal para el BJT, y del comportamiento del diodo en altas frecuencias. Además de contrastar lo simulado con lo teórico, se consideraron valores especificados por los fabricantes de los dispositivos en hojas de datos, contemplando coincidencias y diferencias.
\end{document}