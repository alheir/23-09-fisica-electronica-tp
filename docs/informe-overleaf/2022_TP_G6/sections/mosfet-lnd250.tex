\documentclass[../main.tex]{subfiles}

\begin{document}
El circuito para analizar al dispositivo consiste en una resistencia de drain $R = 7k\Omega$, una fuente de entrada $V_{in} = -0.6$ y una fuente de salida $V_{CC} = 10V$. Simulando, se obtuvo el siguiente punto de trabajo: $Q = (V_{GS_Q} = -0.6V;\, V_{DS_Q}=1.26V;\, I_{DS_Q}=1.25mA)$.    

% Simulando el circuito propuesto por la cátedra en el programa \textit{LTSpice} el punto Q obtenido es el siguiente: $V_{GS} = -0.6\,V$, $V_{DS} = 1.26\,V$ e $I_D = 1,25\,mA$.
    
\subsection{Curvas de entrada y salida}

Al analizar la curva de entrada de la figura \ref{fig:MOSFETd-Plot}, se puede resaltar el hecho de que la tensión umbral ($V_{TH} = -2\,V$) cae dentro de los valores posibles que especifica el fabricante en la hoja de datos del componente ($-3\,V<V_{TH}<-1\,V$) \cite{lnd250}. También resulta fácil ver que el punto Q del circuito está en modo deplexión y que la curva simulada se condice con la teórica.

La curva de entrada también está en gran acuerdo con la estudiada analíticamente ya que la $I_D$ prácticamente deja de variar cuando la $V_{DS}$ varía por encima de cierto valor. Si bien a primera vista puede parecer que el transistor esté operando región de saturación, un análisis más detallado muestra claramente que el punto de operación se encuentra en la zona de triodo.

\subfile{../Plots/MOSFETd-Plot}

\subsection{Parámetros del modelo de pequeña señal}
Se considera el modelo de pequeña señal representado por una conductancia de entrada $g_m$ y una resistencia de salida $r_o$. Los mismos se obtienen de las curvas de entrada y salida en un entorno del punto Q: $g_m = \nicefrac{\partial I_D}{\partial V_{GS}} = 2,57\,\nicefrac{mA}{V}$, y $r_0=\nicefrac{\partial V_{DS}}{\partial I_D}=4635,68\,\Omega$.
\end{document}
%  gd = 0.000215718